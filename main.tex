\documentclass[aspectratio=169]{beamer}
\usetheme{PaloAlto}
\usepackage[sort]{natbib}

% Components of the title page
\logo{\includegraphics[height=0.4in]{logo.png}}



\title[Machine learning e a auditoria]{Machine Learning e a auditoria Contínua}
\setbeamercovered{transparent}

\beamertemplatenavigationsymbolsempty

\begin{document}



\author[M. Alexandre P. C. Junior]{
    \begin{tabular}{c} 
        \Large
        Alexandre Castro\\
        \footnotesize \href{mailto:im.alexandre07@gmail.com}{im.alexandre07@gmail.com}
    \end{tabular}}

\institute{
    \includegraphics[height=0.4in]{logo.png}\\
    \textbf{Semana de Computação}\\
}

\date{22 de outubro de 2020}

\AtBeginSection[]{
    \begin{frame}
        \frametitle{Roteiro}

        \tableofcontents[currentsection]
    \end{frame}
}

\begin{frame}\maketitle\end{frame}


\begin{frame}
    \frametitle{Roteiro}
    \tableofcontents[pausesections]
\end{frame}


\section{Objetivos da Apresentação}
\begin{frame}{Objetivos da Apresentação}
    Os principais objetivos da aula são:
    \begin{itemize}[<+- | uncover@+>]
        \item \textit{`Desmistificar'} machine learning, IA e outros \textit{trend topics}
        \item Conhecer algumas aplicações na auditoria.
        \item Como é o início de uma iniciativa baseada em ML
    \end{itemize}
\end{frame}


\section{Conceitos}
\subsection{Big Data}
\begin{frame}{Big Data}
    \centering
    \includegraphics[height=1in]{tennage_sex.png} \\
    \begin{block}{}
        Big Data:  Fenômeno caracterizado pela geração massiva e 
        ininterrupta de dados, que podem ser processados e armazenados, gerando valor.
    \end{block}
\end{frame}


\begin{frame}{Big Data}
    \centering
    \begin{block}{}
        Os '3+2' \textbf{V}s do Big Data
    \end{block}
    \begin{itemize}
        \item \textbf{V}elocidade;
        \item \textbf{V}olume;
        \item \textbf{V}ariedade;
        \item \alert{\textbf{V}alor};
        \item \alert{\textbf{V}eracidade};
    \end{itemize}
\end{frame}


\subsection{Inteligência Artificial}
\begin{frame}{Inteligência Artificial}
    \centering
    \begin{block}{Diferença entre Machine Learning e IA}
        \begin{itemize}
            \item Se estiver no powerpoint, é IA
            \item No TCC será Machine Learning;
            \item No Jupyter Notebook, regressão linear.
        \end{itemize}
    \end{block}
    \includegraphics[height=1.5in]{ia_powepoint.png}
\end{frame}

\begin{frame}{Inteligência Artificial}
    \centering
    \begin{block}{Definição `formal'}
        Campo de estudo que busca, por meio de ferramentas computacionais, simular
        o comportamento humano, de forma que `a máquina' possa resolver \textbf{problemas}
        da mesma forma que um ser humano resolveria.
    \end{block}
    \begin{figure}
        \includegraphics[height=1.5in]{ai_ml_towardsdatascience.png}
        \caption{\cite{Wu2019} Diagrama de áreas da IA}
    \end{figure}
\end{frame}


\subsection{Machine Learning}
\begin{frame}{Machine Learning}
    \centering
    \includegraphics[height=1.5in]{ultron.jpg}
    \includegraphics[height=1.5in]{math.jpeg}
\end{frame}

\subsection{Auditoria Contínua}
\begin{frame}{Auditoria Contínua}
    \centering
    \includegraphics[height=2.5in]{minority.jpg}
\end{frame}


\begin{frame}{Auditoria Contínua}
    \centering
    \begin{block}{O que é auditoria contínua?}
        \begin{itemize}
            \item Atividade de verificar os atos e processos de gestão em tempo real,
                utilizando bases de dados provenientes dos sistemas operativos da organização.
            \item Visa tratar os erros/desvios antes que sejam efetivados.
            \item Trabalha com indícios, que precisam ser validados pelos auditores
                especialistas de cada área (recursos humanos, finanças, etc.)
        \end{itemize}
    \end{block}
\end{frame}

\section{Aplicações na auditoria}
\subsection{Exemplos}
\begin{frame}
    \frametitle{Aplicações na auditoria}
    \framesubtitle{Aplicação de técnica de clusterização em dados de licitações}
    \begin{itemize}
        \item Objetivo: direcionar os esforços de verificação e controle para aquelas
            situações com maior risco de desvios/erros. 
        \item Empresas do mesmo estado que participaram de 233 pregões juntas.
        \item Identificação de licitações com risco de ocorrência de cartéis.
    \end{itemize}\cite{Silva2020}
\end{frame}


\begin{frame}
    \frametitle{Aplicações na auditoria}
    \framesubtitle{O caso ALICE (TCU)}
    \begin{itemize}
        \item ALICE é um acrônimo para Análise de licitações e editais
        \item O programa realiza uma série de verificações e aponta indícios de
            irregularidades para os auditores do TCU
        \item Esses dados não são em si irregularidades, mas indícios que 
            apontam para o auditor olhar o edital de maneira mais detalhada.
        \item O maior ganho que a gente tem é que os órgãos retiram, anulam ou 
            cancelam os editais e fazem outro da forma correta.
        \item Esse tipo de trabalho poderia ser feito por humanos, mas seria 
            muito custoso porque são, em média, 200 editais por dia
    \end{itemize}
\end{frame}


\subsection{Início da implementação}
\begin{frame}
    \frametitle{Aplicações na auditoria}
    \framesubtitle{Antes disso tudo...}
    \begin{block}{Requisitos para um projeto de ML}
        \begin{itemize}
            \item Profissionais com perfil analítico
            \item Chefia envolvida e com apetite para mudanças(e riscos)
            \item \alert{\textbf{Engenharia de dados}}
            \item Capacitação para todos os profissionais
        \end{itemize}

    \end{block}
\end{frame}


\begin{frame}
    \frametitle{Aplicações na auditoria}
    \framesubtitle{Antes disso tudo...}
    \begin{block}{Exemplo de workflow de auditoria contínua}
        \centering
        \includegraphics[height=2in]{workflow_auditorias.png}
    \end{block}
\end{frame}


\begin{frame}
    \frametitle{Aplicações na auditoria}
    \framesubtitle{Antes disso tudo...}
    \begin{block}{A nossa stack}
        \begin{itemize}
            \item Postgresql
            \item Python
            \item Apache airflow
            \item Caseware IDEA (auditores)
        \end{itemize}
    \end{block}
\end{frame}


\begin{frame}
    \frametitle{Conclusão}
    \centering
        OBRIGADO!\\
        \begin{block}{}
            Contatos:
        \end{block}


    \begin{tabular}{cc}
        Linkedin: & \href{https://www.linkedin.com/in/alexandre-castro-45593415a/}{\includegraphics[height=0.3in]{linkedin.png}}\\
        \hline
        Github:  & \href{https://www.github.com/im-alexandre}{\includegraphics[height=0.3in]{github.png}}\\
        \hline
        E-mail:  &  \href{mailto:im.alexandre07@gmail.com}{\includegraphics[height=0.3in]{gmail.png}}\\
        \hline
    \end{tabular}
\end{frame}

\begin{frame}{Referências}
\bibliographystyle{abbrv}
\bibliography{Bibliografia.bib}
\end{frame}

\end{document}
